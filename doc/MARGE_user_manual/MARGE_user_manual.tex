% MARGE User Manual
%
% Please note this document will be automatically compiled and hosted online
% after each commit to master. Because of this, renaming or moving the
% document should be done carefully. To see the compiled document, go to
% http://planets.ucf.edu/bart-docs/MARGE_user_manual.pdf

\documentclass[letterpaper, 12pt]{article}
\input{top-MARGE_user_manual}

\begin{document}

\begin{titlepage}
\begin{center}

\textsc{\LARGE University of Central Florida}\\[1.5cm]

% Title
\rule{\linewidth}{0.5mm} \\[0.4cm]
{ \huge \bfseries MARGE Users Manual \\[0.4cm] }
\rule{\linewidth}{0.5mm} \\[1.0cm]

\textsc{\Large Machine learning Algorithm for Radiative transfer of Generated Exoplanets}\\[1.5cm]

% Author and supervisor
\noindent
\begin{minipage}{0.4\textwidth}
\begin{flushleft}
\large
\emph{Authors:} \\
Michael D. \textsc{Himes} \\
\end{flushleft}
\end{minipage}%
\begin{minipage}{0.4\textwidth}
\begin{flushright} \large
\emph{Supervisor:} \\
Dr.~Joseph \textsc{Harrington}
\end{flushright}
\end{minipage}
\vfill

% Bottom of the page
{\large \today}

\end{center}
\end{titlepage}

\tableofcontents
\newpage

\section{Team Members}
\label{sec:team}

\begin{itemize}
\item \href{https://github.com/mdhimes/}{Michael Himes}%
  \footnote{https://github.com/mdhimes/}, University of
  Central Florida (mhimes@knights.ucf.edu)
\item Joseph Harrington, University of Central Florida
\item Adam Cobb, University of Oxford
\item David C. Wright, University of Central Florida
\item Zacchaeus Scheffer, University of Central Florida
\end{itemize}

\section{Introduction}
\label{sec:theory}

\noindent This document describes MARGE, the Machine learning Algorithm for Radiative 
transfer of Generated Exoplanets.  MARGE generates exoplanet spectra; processes 
them into a desired format; and trains, validates, and tests neural network (NN)
models to approximate radiative transfer (RT).  At present, MARGE is configured 
to use BART, the Bayesian Atmospheric Radiative Transfer code, for spectra 
generation.

The detailed MARGE code documentation and User Manual\footnote{A compiled PDF 
of the latest version of the manual is available at 
\href{https://planets.ucf.edu/bart-docs/MARGE\_user\_manual.pdf}{https://planets.ucf.edu/bart-docs/MARGE\_user\_manual.pdf}.} 
are provided with the package to assist users in its usage. 
For additional support, contact the lead author (see Section \ref{sec:team}).

MARGE is released under the Reproducible Research Software License.  
For details, see \\
\href{https://planets.ucf.edu/resources/reproducible-research/software-license/}{https://planets.ucf.edu/resources/reproducible-research/software-license/}.
\newline

\noindent The MARGE package is organized as follows: \newline
% The framebox and minipage are necessary because dirtree kills the
% indentation.
\noindent\framebox{\begin{minipage}[t]{0.97\columnwidth}%
\dirtree{%
 .1 MARGE. 
 .2 doc.
 .2 example.
 .2 lib. 
 .3 datagen.
 .4 BART.
 .2 modules. 
 .3 BART. 
}
\end{minipage}}
\vspace{0.7cm}
% \newline is not working here, therefore I use vspace.
% (because dirtree is such a pain in the ass)

\section{Installation}
\label{sec:installation}

\subsection{System Requirements}
\label{sec:requirements}

\noindent MARGE was developed on a Unix/Linux machine using the following 
versions of packages:

\begin{itemize}
\item Python 3.7.2
\item Keras 2.2.4
\item Numpy 1.16.2
\item Matplotlib 3.0.2
\item mpi4py 3.0.3
\item Scipy 1.2.1
\item sklearn 0.20.2
\item Tensorflow 1.13.1
\item CUDA 9.1.85
\item cuDNN 7.5.00
\item ONNX 1.6.0
\item keras2onnx 1.6.1
\item onnx2keras 0.0.18
\end{itemize}

\noindent MARGE also requires a working MPI distribution if using BART for 
data generation.  MARGE was developed using MPICH version 3.3.2.



\subsection{Install and Compile}
\label{sec:install}

\noindent To begin, obtain the latest stable version of MARGE.  Decide on a 
local directory to hold MARGE.  Let the path to this directory be 
MARGE. Now, clone the repository:
\begin{verbatim}
git clone --recursive https://github.com/exosports/MARGE MARGE/
cd MARGE/
\end{verbatim}

\noindent For data generation with BART, users must build BART's submodules:
\begin{verbatim}
make bart
\end{verbatim}

\noindent MARGE contains a file to easily build a conda environment capable of 
executing the software.  Create the environment via
\begin{verbatim}
conda env create -f environment.yml
\end{verbatim}

\noindent Then, activate the environment:
\begin{verbatim}
conda activate marge
\end{verbatim}

\noindent You are now ready to run MARGE.


\section{Example}
\label{sec:example}

\noindent The following script will walk a user through executing all modes of 
MARGE to simulate the emission spectra of an HD 189733 b-like exoplanet with a 
variety of thermal profiles and atmospheric compositions, process the data, and 
train an NN model to quickly approximate spectra over the trained parameter 
space.  These instructions are meant to be executed from a Linux terminal.  

We offer a lightweight example meant to be executed on a machine with 4 cores 
and 8 GB of RAM.  Note that we are compromising the completeness and accuracy 
of the resulting model to ensure that the software can be executed in a 
reasonable amount of time (though it will still take a while!).  To improve 
the results, users may increase the number of spectra generated (numit in BART 
config), use a finer opacity grid (reduce tempdelt and wndelt in BART config), 
increase the number of atmospheric layers (n\_layers in BART config), and/or 
use a more complicated model architecture (more layers, more nodes).

To begin, copy the requisite files to another directory.  Here we 
assume that directory is parallel to MARGE, called run.  From the MARGE 
directory,
\begin{verbatim}
mkdir ../run
cp -a ./example/* ../run/.
cd ../run
\end{verbatim}

\noindent To generate data with BART, a Transit Line-Information (TLI) file 
containing all line list information must be created.  Download the required 
line lists and create the TLI file (this may take a while!):
\begin{verbatim}
./get_lists.sh
../MARGE/modules/BART/modules/transit/pylineread/src/pylineread.py -c pylineread.cfg
\end{verbatim}

\noindent Now, execute MARGE:
\begin{verbatim}
../MARGE/MARGE.py MARGE.cfg
\end{verbatim}

\noindent This will take some time to run.  It will 

\begin{itemize}
\item generate an opacity table,
\item run BART to generate spectra,
\item process the generated spectra into MARGE's desired format,
\item train, validate, and test an NN model, and
\item plot specific predicted vs. true spectra.
\end{itemize}

\noindent Users can disable some steps via boolean flags within the configuration file.  
For details, see the following section.


\section{Program Inputs}
\label{sec:inputs}

The executable MARGE.py is the driver for the MARGE program. It takes a 
a configuration file of parameters.  Once configured, MARGE is executed via 
the terminal as described in Section \ref{sec:example}.


\subsection{MARGE Configuration File}
\label{sec:config}
The MARGE configuration file is the main file that sets the arguments for a 
MARGE run. The arguments follow the format {\ttb argument = value}, where 
{\ttb argument} is any of the possible arguments described below. Note that, 
if generating data via BART, the user must create a BART configuration file 
(see Section \ref{sec:BARTconfig}).

\noindent The available options for a MARGE configuration file are listed below.

\noindent \underline(Directories)
\begin{itemize}
\item inputdir   : str.  Directory containing MARGE inputs.
\item outputdir  : str.  Directory containing MARGE outputs.
\item plotdir    : str.  Directory to save plots. 
                         If relative path, subdirectory within `outputdir`.
\item datadir    : str.  Directory to store generated data. 
                         If relative path, subdirectory within `outputdir`.
\item preddir    : str.  Directory to store validation and test set predictions and true values. 
                         If relative path, subdirectory within `outputdir`.
\end{itemize}

\noindent \underline{Datagen Parameters}
\begin{itemize}
\item datagen    : bool. Determines whether to generate data.
\item datagenfile: str.  File containing functions to generate and/or process 
                         data.  Do NOT include the .py extension!
                         See the files in the lib/datagen directory for examples.
                         User-specified files must have identically-named 
                         functions with an identical set of inputs.  If an 
                         additional input is required, the user must modify the 
                         code in MARGE.py accordingly.  
                         Please submit a pull request if this occurs!
\item cfile      : str.  Name of the configuration file for data generation.
                         Can be absolute path, or relative path to `inputdir`.
\item processdat : bool. Determines whether to process the generated data.
\item preservedat: bool. Determines whether to preserve the unprocessed data after 
                   completing data processing.
                   Note: if False, it will PERMANENTLY DELETE the original, 
                         unprocessed data!
\end{itemize}


\noindent \underline{Neural Network (NN) Parameters}
\begin{itemize}
\item nnmodel    : bool. Determines whether to use an NN model.
\item resume     : bool. Determines whether to resume training a previously-trained 
                   model.
\item seed       : int.  Random seed.
\item trainflag  : bool. Determines whether to train    an NN model.
\item validflag  : bool. Determines whether to validate an NN model.
\item testflag   : bool. Determines whether to test     an NN model.

\item TFR\_file  : str.  Prefix for the TFRecords files to be created.
\item buffer     : int.  Number of batches to pre-load into memory.
\item ncores     : int.  Number of CPU cores to use to load the data in parallel.

\item normalize  : bool. Determines whether to normalize the data by its mean and 
                   standard deviation.
\item scale      : bool. Determines whether to scale the data to be within a range.
\item scalelims  : floats. The min, max of the range of the scaled data.
                     It is recommended to use -1, 1

\item weight\_file: str.  File containing NN model weights.
                          NOTE: MUST end in .h5
\item input\_dim  : int.  Dimensionality of the input  to the NN.
\item output\_dim : int.  Dimensionality of the output of the NN.
\item ilog        : bool. Determines whether to take the log10 of the input  data.
\item olog        : bool. Determines whether to take the log10 of the output data.

\item gridsearch : bool. Determines whether to perform a gridsearch over 
                         architectures.
\item architectures: strings. Name/identifier for a given model architecture.  
                         Names must not include spaces!
                         For multiple architectures, separate with a space or 
                         indented newlines.
\item nodes : ints. Number of nodes per layer with nodes.
\item layers: strings. Type of each hidden layer.  
                       Options: dense, conv1d, maxpool1d, avgpool1d, flatten, 
                                dropout
\item lay\_params: list. Parameters for each layer (e.g., kernel size). 
                        For no parameter or the default, use None.
\item activations: strings. Type of activation function per layer with nodes.
                        Options: linear, relu, leakyrelu, elu, tanh, sigmoid, 
                                 exponential, softsign, softplus, softmax
\item act\_params: list. Parameters for each activation.  
                        Use None for no parameter or the default value.
                        Values specified only apply to LeakyReLU and ELU.

\item epochs     : int.  Maximum number of iterations through the training data set.
\item patience   : int.  Early-stopping criteria; stops training after `patience` 
                   epochs of no improvement in validation loss.
\item batch\_size : int.  Mini-batch size for training/validation steps.

\item lengthscale: float. Minimum learning rate.
\item max\_lr     : float. Maximum learning rate.
\item clr\_mode   : str.   Specifies the function to use for a cyclical learning rate 
                    (CLR).
                    'triangular' linearly increases from `lengthscale` to 
                    `max\_lr` over `clr\_steps` iterations, then decreases.
                    'triangular2' performs similar to `triangular', except that 
                    the `max\_lr` value is decreased by 2 every complete cycle,
                    i.e., 2 * `clr\_steps`.
                    'exp\_range' performs similar to 'triangular2', except that 
                    the amplitude decreases according to an exponential based 
                    on the epoch number, rather than the CLR cycle.
\item clr\_steps  : int.   Number of steps through a half-cycle of the learning rate.
                    E.g., if using clr\_mode = 'triangular' and clr\_steps = 4, 
                    Every 8 epochs will have the same learning rate.
                    It is recommended to use an even value.
                    For more details, see Smith (2015), Cyclical Learning Rates 
                    for Training Neural Networks.
\end{itemize}

\noindent \underline{Plotting Parameters}
\begin{itemize}
\item xvals       : str.  .NPY file with the x-axis values corresponding to 
                          the NN output.
\item xlabel      : str.  X-axis label for plots.
\item ylabel      : str.  Y-axis label for plots.
\item plot\_cases : ints. Specifies which cases in the test set should be 
                   plotted vs. the true spectrum.
                   Note: must be separated by spaces or indented newlines.
\end{itemize}

\noindent \underline{Statistics Files}
\begin{itemize}
\item fmean      : str.  File name containing the mean of each input/output.
                         Assumed to be in `inputdir`.
\item fstdev     : str.  File name containing the standard deviation of each 
                   input/output.
                         Assumed to be in `inputdir`.
\item fmin       : str.  File name containing the minimum of each input/output.
                         Assumed to be in `inputdir`.
\item fmax       : str.  File name containing the maximum of each input/output.
                         Assumed to be in `inputdir`.
\item rmse\_file  : str.  Prefix for the file to be saved containing the root mean 
                   squared error of predictions on the validation \& test data.
                   Saved into `outputdir`.
\item r2\_file    : str.  Prefix for the file to be saved containing the 
                   coefficient of determination (R\^2) of predictions on the 
                   validation \& test data.
                   Saved into `outputdir`.
\item filters : strings. (optional) Paths/to/filter files that define a 
                   bandpass over `xvals`. 
                   Two columns, wavelength and transmission.
                   Used to compute statistics for band-integrated values.
\item filt2um : float. (default: 1.0) Factor to convert the filter files' 
                       X values to microns. 
                       Only used if `filters` is specified.
\end{itemize}



\subsubsection{BART Configuration File}
\label{sec:BARTconfig}

The BART User Manual details the creation of a BART configuration file.  For 
compatibility with MARGE, users must ensure two specific arguments are set 
within the configuration file:
\begin{itemize}
\item savemodel: base file name of the generated data. MUST have '.npy' file 
                 extension.
\item modelper: an integer that sets the batch size of each `savemodel` file.
\end{itemize}

\noindent Note that `modelper` batch size corresponds to the iterations per 
chain.  For example, if using 10 parallel chains, a `modelper` of 512 would 
save out files of 5120 spectra each.

\noindent  Executing BART requires a Transit Line-Information (TLI) file to 
be created.  For details on generating a TLI file, see the Transit User Manual.
For an example, see Section \ref{sec:example}.



\section{Program Outputs}
\label{sec:outputs}

MARGE produces the following outputs if all modes are executed:

\begin{itemize}
\item simulated spectra.npy files
\item processed spectra.npy files
\item .NPY file of the mean of training set inputs and outputs
\item .NPY file of the standard deviation of training set inputs and outputs
\item .NPY file of the minima of training set inputs and outputs
\item .NPY file of the maxima of training set inputs and outputs
\item .NPY file of the size of the training, validation, and test sets
\item TFRecords files of the data set
\item file containing the NN model and weights
\item predicted and true spectra
\item RMSE and R\^2 statistics
\item plots of true and predicted spectra
\end{itemize}

\noindent Note that BART's output files are not discussed here; see the BART 
User Manual for details.



\section{Be Kind}
\label{sec:bekind}
Please cite this paper if you found this package useful for your research:

\begin{itemize}
\item Himes et al. (2020), submitted to PSJ.
\end{itemize}

\begin{verbatim}
@article{HimesEtal2020psjMARGEHOMER,
   author = {{Himes}, Michael D. and {Harrington}, Joseph and {Cobb}, Adam D. and {G{\"u}ne{\textcommabelow s} Baydin}, At{\i}l{\i}m and {Soboczenski}, Frank and
         {O'Beirne}, Molly D. and {Zorzan}, Simone and
         {Wright}, David C. and {Scheffer}, Zacchaeus and
         {Domagal-Goldman}, Shawn D. and {Arney}, Giada N.},
    title = "Accurate Machine Learning Atmospheric Retrieval via a Neural Network Surrogate Model for Radiative Transfer",
  journal = {PSJ},
     year = 2020,
    pages = {submitted to PSJ}
}
\end{verbatim}

\noindent Thanks!

% \section{Further Reading}
% \label{sec:furtherreading}

% TBD: Add papers here.


\end{document}
